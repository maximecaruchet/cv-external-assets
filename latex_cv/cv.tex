%!TEX TS-program = xelatex
\documentclass[]{friggeri-cv}
\addbibresource{bibliography.bib}

\begin{document}
\header{Maxime}{Caruchet}
       {Ingénieur chez Niji}


% In the aside, each new line forces a line break
\begin{aside}
  \section{À propos}
    \includegraphics[width=\textwidth]{personal_details}
    ~
    \includegraphics[width=\textwidth]{personal_details3}
  \section{Langues}
    Anglais \& Espagnol
  \section{Compétences devops}
    Python, Perl, Linux
    Ansible, Consul
    LXD, Docker, VirtualBox
    Jenkins, Protractor, Cucumber
  \section{Compétences Web}
    AngularJS, Node.js
    TypeScript, Javascript
    Drupal, PHP
    HTML5, CSS3
  \section{Autres compétences}
    Git
    VMware, pfSense, OpenVPN
    Lua, Java, C
    MySQL
\end{aside}

\section{Expériences}

\begin{entrylist}
  \entry
    {10/15-présent}
    { Niji, Cesson-Sévigné}
    {Alternant puis ingénieur}
    {\emph{Maintenance sur les outils DevOps existants}
    \begin{itemize}
      \item Maintenance d'un \textbf{orchestrateur} développé en interne
      \item Intégration continue avec \textbf{Jenkins}
      \item Provisioning avec \textbf{Ansible}
    \end{itemize}
    \emph{Développement sur le portail de consommation d'EDF (e.quilibre)}
    \begin{itemize}
      \item Développement d'évolutions et maintenance du portail
      \item Développement \textbf{PHP} avec le CMS \textbf{Drupal}
      \item Utilisation de \textbf{Behat} pour tester le portail en intégration continue
    \end{itemize}
    \emph{Refonte du portail e.quilibre}
    \begin{itemize}
      \item Développement en \textbf{AngularJS}
      \item Mise en place de tests end-to-end avec \textbf{Protractor} et \textbf{Cucumber.js}
    \end{itemize}}
  \entry
    {06/15-10/15}
    {Niji, Cesson-Sévigné}
    {Stage M1}
    {\emph{Apprentissage de la démarche DevOps}
    \begin{itemize}
      \item Développement d'un \textbf{orchestrateur} en \textbf{Python}
      \item Aide à la mise en place de projets utilisant l'orchestrateur interne
    \end{itemize}
    \emph{Développement sur le portail de consommation d'EDF (e.quilibre)}
    \begin{itemize}
      \item Développement et maintenance du portail
      \item Développement \textbf{PHP} avec le CMS \textbf{Drupal}
    \end{itemize}}
  \entry
    {2015}
    {ISEN Brest}
    {Projet M1}
    {\emph{Plateforme de gestion de machines virtuelles}
    \begin{itemize}
      \item Développement en \textbf{PHP}
      \item Utilisation de l'API d'OVH pour la gestion des machines
    \end{itemize}}
  \entry
    {2014}
    {ISEN Brest}
    {Projet 1ère anée ISEN Brest}
    {\emph{Banc de régulation thermique}
    \begin{itemize}
      \item Conception d'une carte électronique
      \item Développement d'un programme VHDL en interaction avec cette carte
      \item Développement d'un programme en \textbf{C} en interaction avec cette carte
    \end{itemize}}
  \entry
    {06/13-présent}
    { Gestion serveur}
    {Temps personnel}
    {\emph{Administrateur système d'un serveur dédié}
    \begin{itemize}
      \item Installation de serveurs (jeu, vocal, web)
      \item Administration : utilisation de \textbf{Linux}, \textbf{VMware ESXi}, \textbf{pfSense}
      \item Automatisation de l'administration avec \textbf{Bash}, \textbf{Python}, \textbf{Docker}
    \end{itemize}}
\end{entrylist}

\section{Formation}

\begin{entrylist}
  \entry
    {2013-2016}
    {Études d'ingénieur généraliste}
    {ISEN Brest}
    {Option génie logiciel}
  \entry
    {2011-2013}
    {CPGE}
    {lycée Victor Grignard, Cherbourg-Octeville}
    {Classes préparatoires aux grandes écoles MPSI/MP}
  \entry
    {2011}
    {Baccalauréat S}
    {lycée Jean-François Millet, Cherbourg-Octeville}
    {Spécialité mathématiques}
\end{entrylist}

\section{Centres d'intérêt}

Batterie, informatique et technologie, badminton, squash

\end{document}
